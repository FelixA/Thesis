%---------------------------------------------------------------------------------------------------
% Zusammenfassung
%---------------------------------------------------------------------------------------------------
% addchap* ist n KOMA Befehl. Er fügt eine Chapterüberschrift dem Dok hinzu, aber es taucht nicht im Inhaltsverzeichnis auf, und auch nicht als klickable Link im pdf
\addchap*{Kurzfassung}
\vspace{1cm}
%This paper gives an overview -> über was
%Der Titel der Thesis sollte nach der Kurzfassung vollständig verstanden sein. 
%Stichworte der Kapitel müssen hier genannt werden. Als optimal gilt: \\
%+ 2 Sätze zur Einleitung \\
%+ 2 Sätze zur Problemstellung \\
%+ 2 Sätze zur Durchführung \\
%+ 2 Sätze zum Ergebniss \\
%Insgesamt sollte Bereich nicht größer als 200-300 Wörter werden (max. halbe Seite DINA4) und am Besten zum Schluss verfasst werden. \\
%Eine ......  hat die und die Ergnisse erzielt... 


Traditionelles Geschäftsprozessmanagement hat die Annahme als Grundlage, dass es immer einen idealen Weg gibt, Prozesse so zu gestalten, dass sie möglichst schnell und effizient nach einem bewährten Vorgehen durchgeführt werden können. Dafür werden Experten eingesetzt, die Unternehmensprozesse analysieren und basierend auf diesen Erkenntnissen einen idealen Prozess definieren. Solche Standardprozesse vereinfachen die Messung von \ac{KPI} sowie die Überprüfung der Einhaltung von Compliance.

Die Praxis zeigt jedoch, dass viele Arbeiten spezifisches Wissen der Mitarbeiter voraussetzen. Diese müssen die Möglichkeit haben, flexibel auf neue Informationen oder geänderte Bedingungen zu reagieren. Solche Arbeiten lassen sich nicht vorab definieren, da schlichtweg nicht genügend Informationen vorhanden sind, um alle Möglichkeiten des Prozessablaufes abzudecken. 

\ac{ACM} ist ein Ansatz, diese Arbeiter in der Ausübung ihrer Tätigkeiten optimal zu unterstützen. Dabei hat die Person, die den Prozess ausführt, die Möglichkeit, diesen zur Laufzeit zu definieren und zu modifizieren. Die Übertragung der Prozessverantwortung an den ausführenden Mitarbeiter eröffnet ihm jedoch die Möglichkeit, Unternehmensrichtlinien zu umgehen. 

Welche Möglichkeiten es gibt, trotzdem Unternehmensrichtlinien durchzusetzen, und wie weit sich diese mit der \ac{ACM} Definition vereinbaren lassen, wird in dieser Arbeit untersucht.
Neben den theoretischen Möglichkeiten wird ein System aufgesetzt um zu überprüfen, ob diese in der Praxis einsetzbar sind. Dafür wird als Referenzmodell das Vorgehen \ac{WCS} genutzt, um den Vertrieb in der Ausführung seiner Tätigkeiten optimal zu unterstützen.
Das Ergebnis der theoretischen Untersuchung ist eine Aufstellung, das diese Möglichkeiten darstellt und in das Modell der \glqq Seven Domains of Predictability\grqq{} einordnet.

Der zunehmende Wettbewerbsdruck zwischen Unternehmen führt dazu, dass Flexibilität und Reaktionszeit zum entscheidenden Wettbewerbsfaktor werden. Das Geschäftsprozessmanagement ist die Disziplin, Geschäftsprozesse so zu organisieren, dass schnell auf veränderte Rahmenbedingungen reagiert werden kann.

\smallskip\noindent Softwarehersteller greifen diesen Ansatz auf, und bieten Produkte, die das Geschäftsprozessmanagement unterstützen sollen. Die hohe Komplexität erschwert eine Auswahl und bindet ein Unternehmen lange an ein eingeführtes System. Erschwerend kommt hinzu, dass am Markt sehr viele Systeme, mit teilweise großen Unterschieden angeboten werden. Zusätzlich stehen mehrere Open Source Systeme zur Wahl.
%Ziel -> Tauglichkeit BPM -> Open Source Alternative

\vfill