\cleardoublepage
\addcontentsline{toc}{section}{Abstract}
\addchap*{Abstract}
\vspace{1cm}

Satoshi Nakamoto beschreibt \gls{Bitcoin} als ein Peer-to-Peer Zahlungssystem \cite{Nakamoto.31.10.2008}. Dabei wurde eine Vielzahl verschiedener, möglicher Einsatzzwecke der zugrundeliegenden Blockchain-Technologie vernachlässigt. In dieser Arbeit wird gezeigt, dass ein dezentrales Zahlungsmittel nur ein möglicher Einsatzzweck für die eigentliche Innovation der dezentralen, Vertrauenslosen Konsens-Erzielung ist. 
Während klassischerweise für die Übertragung von Werten zwischen zwei Parteien ein vertrauenswürdiger Mittelsmann benötigt wurde - beispielsweise eine Bank zur Kontenführung oder eine Abwicklungsgesellschaft zur Abwicklung von Wertpapiergeschäften - kann eine solche Instanz durch Blockchain-Technologie eliminiert werden. Die Vorteile eines dezentrales Konsens bestehen aus einer anonymen, transparenten, nachvollziehbaren, effizienteren und kostengünstigeren Abwicklung von Wertübertragungen. \\
Anwendungen, die auf Basis der Blockchain-Technologie basieren, werden als smarte Verträge bezeichnet, mit der Definition einer Crypto-Währung als einer der grundlegenden Verträge. In dieser Arbeit wird gezeigt, dass smarte Verträge einen disruptiven Charakter haben, der neben klassischen Anwendungen zur Wertübertragung ohne Bedarf eines Mittelmanns, völlig neue Anwendungszwecke bis hin zu innovativen Unternehmensorganisationen erlauben. \\
Dazu wird einerseits die Blockchain Technologie sowie ihre historische Entwicklung bis hin zu einer turing-vollständigen Technologie erläutert und auf technische Herausforderungen geprüft. Andererseits wird eine Auswahl möglicher Anwendungsszenarien vorgestellt, gewertet und auf ihre fachliche Umsetzbarkeit geprüft.\\
Die Ergebnisse zeigen, ... 
