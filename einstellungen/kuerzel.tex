% Hier sind Makros definiert, die die Eingabe erleichtern sollen. F�r korrekte Abst�nde zwischen
% "z.B." sorgt also ein "z.\,B." (LaTeX-Befehl f�r kleineren Abstand)
% Schneller schreibt sich das durch das Makro "\zB":

% \newcommand{\zB}{z.\,B.\ }

% Hier ist der Rest aber mit dem Paket xspace verwirklicht. Damit kann
% man bei Bedarf den Abstand mit "\hspace" exakt eingeben. Dann zeigt
% LaTeX keine Toleranz bei den Abk�rzungen und macht eben exakt das
% untenstehende. 

%\renewcommand{\entryname}{K\"urzel}
%\renewcommand{\descriptionname}{Beschreibung}

\newcommand{\eu}{\@\xspace\officialeuro\hspace{0.125em}\@\xspace}
\newcommand{\vgl}{vgl.\@\xspace} 
\newcommand{\abb}{Abb.\@\xspace} 
\newcommand{\tab}{Tab.\@\xspace} 
\newcommand{\zb}{z.\nolinebreak[4]\hspace{0.125em}\nolinebreak[4]B.\@\xspace}
\newcommand{\bzw}{bzw.\@\xspace}
\newcommand{\dahe}{d.\nolinebreak[4]\hspace{0.125em}h.\nolinebreak[4]\@\xspace}
\newcommand{\etc}{etc.\@\xspace}
\newcommand{\s}{S.\@\xspace}
\newcommand{\bzgl}{bzgl.\@\xspace}
\newcommand{\so}{s.\nolinebreak[4]\hspace{0.125em}\nolinebreak[4]o.\@\xspace}
\newcommand{\ia}{i.\nolinebreak[4]\hspace{0.125em}\nolinebreak[4]A.\@\xspace}
\newcommand{\sa}{s.\nolinebreak[4]\hspace{0.125em}\nolinebreak[4]a.\@\xspace}
\newcommand{\su}{s.\nolinebreak[4]\hspace{0.125em}\nolinebreak[4]u.\@\xspace}
\newcommand{\ua}{u.\nolinebreak[4]\hspace{0.125em}\nolinebreak[4]a.\@\xspace}
\newcommand{\og}{o.\nolinebreak[4]\hspace{0.125em}\nolinebreak[4]g.\@\xspace}

\newcommand{\prof}{Prof.\@\xspace}
\newcommand{\dr}{Dr.\@\xspace}

\newcommand{\HSKA}{\myHochschulename \xspace}
\newcommand{\GNU}{GNU\xspace}
\newcommand{\GPL}{\GNU Public License\xspace}

\newcommand{\hpops}{HP Operations Orchestration}

\newcommand{\ACM}{ACM\xspace}
\newcommand{\PDA}{PDA\xspace}

\newcommand{\myLatex}{{\rmfamily\LaTeX\xspace}}		% LaTeX in eigenem Font