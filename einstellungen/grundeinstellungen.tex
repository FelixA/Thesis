%---------------------------------------------------------------------------------------------------
% Einstellungen
% (gelten nur in Zusammenarbeit mit pdflatex)
%---------------------------------------------------------------------------------------------------
\documentclass[
  pagesize,	                                           % flexible Auswahl des Papierformats
  BCOR12mm,      	                                     % Bindungskorrektur
  DIV12,																						% Konstruktion des Satzspiegels, abhängig von der verwendeten Zeichengröße zu wählen; gute Lesbarkeit, wenn eine Zeile etwa 66-72 Zeichen 
  headsepline,                                         % Strich unter der Kopfzeile
  12pt,                                                % 12pt Schriftgröße
	liststotoc,																					 % Tabellen- und Abbildungsverzeichnis im 																																 			 % Inhaltsverzeichnis
	idxtotoc,																						 % Index im Inhaltsverzeichnis	
  bibtotoc,                                            % Literaturverzeichnis im Inhaltsverzeichnis  
  twoside,
	openright,% openany openright                        % Neue Kapitel beginnen auf nächster leeren rechten Seite
	notitlepage,%titlepage, notitlepage		 								 % selects if there should be a separate title page
]{scrbook}%scrreprt}                                            % KOMA-Scriptklasse Report

% DIN A4
\KOMAoptions{paper=a4}

% true= zweiseitiger Druck
% false= einseitiger Druck
\KOMAoptions{twoside=true}

% 12pt Schriftgröße
%\KOMAoptions{fontsize=12pt}

% Europäischer Satz: Abstand zwischen Absätzen
%\KOMAoptions{parskip=half}

% Spezielle Formatierung, die erlaubt, dass die Zusammenfassung vor dem Inhaltsverzeichnis steht
%\KOMAoptions{abstract=true}

% false= Es handelt sich um die endgültige Version
% true= Es handelt sich um eine Vorabversion	
\KOMAoptions{draft=false}

%---------------------------------------------------------------------------------------------------
\usepackage{todonotes}
\usepackage[english,ngerman]{babel}                    % deutsche Trennmuster
\usepackage[TS1,T1]{fontenc}                               % EC-Schriften, Trennstellen nach Umlauten
\usepackage[utf8]{inputenc}                          % direkte Umlauteingabe (ä statt "a)
                                                       % latin1/latin9 für unixoide Systeme
                                                       % (latin1 ist auch unter Win verwendbar)
                                                       % ansinew für Windows
                                                       % applemac Macs
                                                       % cp850 OS/2
%TODO: Serifenschrift als default Schrift setzen
\usepackage{times}              											 % Schriften Paket
\usepackage[official]{eurosym} 
\usepackage{mathptmx}
\usepackage{array,ragged2e} 													 % Wichtig für Abstandsformatierung
\usepackage{color}
\usepackage{float}
\usepackage{harveyballs}
\usepackage[xindy,toc]{glossaries}
\usepackage{natbib}
\usepackage[citeonce*]{footbib}
\usepackage{url}
\renewcommand{\harvardurl}[1]{\textbf{URL:} \url{#1}}
%\usepackage{bigfoot}
\makeglossaries
%---------------------------------------------------------------------------------------------------
%\usepackage{cmbright}                                  % serifenlose Schrift als Standard
                                                       % + alle für TeX benötigten mathematischen
                                                       %   Schriften einschließlich der AMS-Symbole
%\usepackage[scaled=.90]{helvet}                        % skalierte Helvetica als \sfdefault
%\renewcommand{\familydefault}{\rmfamily}								% als Standardschrift setzen

%\usepackage{courier}                                   % Courier als \ttdefault
%\renewcommand{\familydefault}{\ttdefault}

%---------------------------------------------------------------------------------------------------
\usepackage[automark]{scrpage2}                        % Anpassung der Kopf- und Fußzeilen
\usepackage{xspace}                                    % Korrekter Leerraum nach Befehlsdefinitionen
%\usepackage[onehalfspacing]{setspace}
%\usepackage{setspace}																	 % Dieses Package brauchen wir für den 				
																											 % anderthalbzeiligen Abstand.
%\onehalfspacing															% \singlespacing \doublespacing
%\setlength{\headheight}{1.1\baselineskip}		

%\usepackage[numbers]{natbib}                                    % Neuimplementierung des \cite-Kommandos
%\usepackage[square]{natbib}
%\usepackage[square]{natbib}                                  % Neuimplementierung des \cite-Kommandos
%\usepackage[]{natbib}
%\usepackage{biblatex}
%\usepackage{bibgerm}       											       % Deutsche Bezeichnungen
%\usepackage{cite}
%  \renewcommand{\citeleft}{(}	%runde Klammer links
%  \renewcommand{\citeright}{)}	%runde Klammer rechts

\usepackage[absolute]{textpos}                         % placing boxes at absolute positions
\usepackage[final]{pdfpages}                           % include pages of external PDF documents
\usepackage{listings}
\usepackage{paralist}                                  % Spaltenbreite bis zur Seitenbreite dehnen
\usepackage{makeidx}																	 % Paket zur Erstellung eines Stichwortverzeichnisses
\makeindex	
\usepackage[autostyle]{csquotes} 	
																				 % Automatische Erstellung des Stichwortverzeichnis
\usepackage[intoc,
						german,
						prefix]{nomencl}

%---------------------------------------------------------------------------------------------------
% Tabelle
%---------------------------------------------------------------------------------------------------
\usepackage{tabularx}                          %Tabellenspalten zusammenlegen
\usepackage{multirow}                          %Tabellezeilen zusammenlegen
\usepackage{booktabs}                                   %schöne Tabellen
\usepackage{longtable}                        %mehrseitige Tabellen
\usepackage{rotating}                           %vertikale Tabellenspaltenausrichtung
\usepackage{array}   
\usepackage{colortbl} % Paket für die Farbe 
\usepackage{xcolor} % Paket für farbige Texte
\usepackage[justification=centering,font=small,labelfont=bf,format=plain,up,textfont=it,up]{caption}
\usepackage{bigstrut}
\usepackage{subcaption}
\usepackage{lipsum}
\usepackage{tabu}
\usepackage{pdflscape}
\usepackage{graphicx}
\usepackage{enumitem}
\usepackage{hhline}
\usepackage{ragged2e}
\usepackage{setspace} %zeilenabstand in Tabelle
\definecolor{Gray}{gray}{0.9}
\definecolor{White}{rgb}{1.0,1.0,1.0}
\newcolumntype{P}[1]{>{RaggedRight\arraybackslash}p{#1}}

%---------------------------------------------------------------------------------------------------
% Abkürzungen
%---------------------------------------------------------------------------------------------------
\usepackage[
printonlyused                           %nur benutzte im Verzeichnis anzeigen
]{acronym}                                           %Paket für Abkürzungen, withpage Option für Seitennummer
%---------------------------------------------------------------------------------------------------





%---------------------------------------------------------------------------------------------------
% Neue Befehle
%---------------------------------------------------------------------------------------------------
%---------------------------------------------------------------------------------------------------
% Neue Befehle
%---------------------------------------------------------------------------------------------------

\newcommand{\myCr}{\linebreak}
\newcommand{\abstandBeispiel}{\textheigth}
\newcommand{\myName}{Ingo Foth}
\newcommand{\myTitel}{Komplexe Entscheidungsfindung innerhalb eines Workflows durch Integration eines Regelwerks am Beispiel eines Anwendungsfalls aus dem IT-Management Bereich}
\newcommand{\myLocation}{Stuttgart}
\newcommand{\myHochschulename}{Hochschule Karlsruhe Technik und Wirtschaft}
\newcommand{\myFakultaetname}{Fakultät Informatik und Wirtschaftsinformatik}
\newcommand{\myStudiengang}{Fachgebiet Wirtschaftsinformatik}


% Neue Befehle um Headerdarstellung zu verändern
%---------------------------------------------------------------------------------------------------
\newcommand{\chapterHeaderWithNumbers}{
	\clearscrheadings
	\clearscrplain
	\clearscrheadfoot
	% laesst Seitenzahl oben rechts auch bei Chapter erscheinen
	\ihead[\pagemark]{\pagemark}
	\ohead{\headmark}
}

\newcommand{\chapterHeaderWithoutNumbers}{
	\clearscrheadings
	\clearscrplain
	\clearscrheadfoot
	% verhindert eine Seitenzahl oben rechts bei Chapter
	\ihead{\headmark}
	\ohead{\pagemark}
}

%---------------------------------------------------------------------------------------------------
%% The \cite command functions as follows:
%%   \cite{key}                ==>>  Jones u. a. (1990)
%%   \cite[chap. 2]{key}       ==>>  (Jones u. a. 1990, chap. 2)
%%   \cite[e.g.][]{key}        ==>>  (e.g. Jones u. a. 1990)
%%   \cite[e.g.][p. 32]{key}   ==>>  (e.g. Jones u. a. p. 32)
%%   \citep{key}               ==>>  (Jones u. a. 1990)
%%   \citep*{key}              ==>>  (Jones, Baker und Smith 1990)
%%   \citet{key}               ==>>  Jones u. a. (1990)
%%   \citet*{key}              ==>>  Jones, Baker und Smith (1990)
%%   \citeauthor{key}          ==>>  Jones u. a.
%%   \citeauthor*{key}         ==>>  Jones, Baker und Smith
%%   \citeyear{key}            ==>>  1990
%---------------------------------------------------------------------------------------------------
% Befehl zum Erstellen und Hervorheben eines Zitats
% Parameter:
% 1. Zitat
% 2. Author
% 3. Quelle
%---------------------------------------------------------------------------------------------------
\newcommand{\myCitation}[3]{
	%\begin{flushright}
	\begin{center}
	\begin{minipage}{.9\linewidth}
	\footnotesize\rmfamily\itshape
		\footnotesize\rmfamily\itshape#1
		\vspace{.005cm}
		\begin{flushright}
		#2
		\end{flushright}
		\vspace{.1cm} 
		#3
	\end{minipage}
	\end{center}
	%\end{flushright}
	\nobreakspace
}

%---------------------------------------------------------------------------------------------------
% Der folgende Befehle wurden aus der Vorlage von Michael Knop übernommen
%--------------------------------------------------------------------------------------------------
%---------------------------------------------------------------------------------------------------
% Ident
%---------------------------------------------------------------------------------------------------
\newcommand{\ident}[1]{                             % ein Parameter
	\small\ttfamily#1\sffamily\normalsize
}

%---------------------------------------------------------------------------------------------------
% Umbenennen des Symbolverzeichnisses
%---------------------------------------------------------------------------------------------------
\renewcommand{\nomname}{Glossar}				% Das Symbolverzeichnis heisst nun "Glossar"
\renewcommand{\nomlabel}[1]{						% Die zu erklärenden Begriffe sind nun fett hervorgehoben
	\hfil \textbf{#1} \hfil
}

%---------------------------------------------------------------------------------------------------
% Ein paar ganz nützliche Befehle von Lars Mählmann
%---------------------------------------------------------------------------------------------------
%für Kommentare 
\newcommand{\colb}{\color{green}}
\newcommand{\colbl}{\color{black}}

%---------------------------------------------------------------------------------------------------
% Fügt ein Wort dem Index zu
%---------------------------------------------------------------------------------------------------
\newcommand{\toIndex}[1]{#1\index{#1}}

%---------------------------------------------------------------------------------------------------
% Befehle zum Erstellen des Index
% \addIndexEntry{Eintrag in den Index}
% \addSubIndexEntry{Eintrag in den Index}{Eintrag des übergeordneten Eintrags}
%---------------------------------------------------------------------------------------------------
\newcommand{\addIndexEntry}[1]{#1\index{#1}}
\newcommand{\addSubIndexEntry}[2]{#1\index{#2!#1}}

%---------------------------------------------------------------------------------------------------
% Erstellung von Deckblatt (Seite 1) und Titelblatt (Seite 2)
%---------------------------------------------------------------------------------------------------
\newcommand{\createCoverAndTitlePage}[8]{
	\createCover{#1}{#2}{#3}
	\createTitlePage{#1}{#2}{#3}{#4}{#5}{#6}{#7}{#8}
}

%---------------------------------------------------------------------------------------------------
% Erstellung von Deckblatt (Seite 1) 
% Anwendung:
% \createCover{Art der Arbeit}{Autor}{Titel}
%---------------------------------------------------------------------------------------------------
\newcommand{\createCover}[3]{
	\thispagestyle{empty}
	\begin{titlepage}

	\setlength{\TPHorizModule}{1mm}
	\setlength{\TPVertModule}{1mm}
	\textblockorigin{0mm}{0mm} % start everything near the top-left corner

	% Art der Arbeit
	\begin{textblock}{111}(83,115)
		\begin{minipage}[c][1,78cm][c]{11,09cm}		
  		\fontsize{22pt}{20pt}
  		\selectfont
  		\begin{center}
  		#1arbeit
  		\end{center}
		\end{minipage}
	\end{textblock}

	% Name & Titel
	\begin{textblock}{111}(83,131)
		\begin{minipage}[c][4,81cm][t]{11,09cm}	
		\linespread{1.2}	
    \fontsize{16pt}{14pt}    
    \selectfont
    \begin{center}
    #2 \\ \medskip
    #3
    \end{center}
    \end{minipage}
	\end{textblock}
	% Infos zur Arbeit und zum Fachbereich
	\begin{textblock}{186}(22,264)
  	\begin{minipage}[t][5,72cm][l]{17,57cm}
    	\fontsize{12pt}{12pt}
    	\selectfont
			{\em \myFakultaetname}
  	\end{minipage}
	\end{textblock}
	\newpage
	\end{titlepage}
%---------------------------------------------------------------------------------------------------
% Wichtig! Entsprechendes Auskommentieren!
%---------------------------------------------------------------------------------------------------
  %\includepdf{pdf/titel}           				% zum Ausdruck auf blanko Papier
  %\includepdf{pdf/titel_leer}           	% zum Ausdruck auf die Pappe
  
  % TODO Deckblatt einscannen und in elektrische Abgabe integrieren
  \includepdf{pdf/titel_hska}							% evtl. zur elektron. Abgabe als pdf
}

%---------------------------------------------------------------------------------------------------
% Erstellung von Titelblatt (Seite 2) 
% Anwendung:
% \createTitlePage{Art der Arbeit}{Author}{Titel}{Studiengang}{Erstprüfer}{Zweitprüfer}
%---------------------------------------------------------------------------------------------------
\newcommand{\createTitlePage}[7]{
	\thispagestyle{empty}

	\setlength{\TPHorizModule}{1mm}
	\setlength{\TPVertModule}{\TPHorizModule}
	\textblockorigin{0mm}{0mm} % start everything near the top-left corner

	% Name & Titel
	\begin{textblock}{130}(40,63)
		\begin{minipage}[c][5,9cm][t]{13cm}
			\begin{center}
			\linespread{1.2}
			\fontsize{18pt}{18pt}
  		\selectfont
  		#2 \\ \medskip
  		\fontsize{16pt}{16pt}
  		#3
  		\end{center}
		\end{minipage}  	
	\end{textblock}

	% Infos zur Arbeit und zum Fachbereich
	\begin{textblock}{126}(32,214)
  	\begin{minipage}[t][5,72cm][l]{12,57cm}
    	\fontsize{12pt}{12pt}
    	\selectfont
    	#1arbeit eingereicht im Rahmen der #1prüfung\\
    	im #4\\
			%am Department Wirtschaftsinformatik\\
			der \myFakultaetname \\
			der \myHochschulename \\
			\\\
			\\\
			Betreuender Prüfer : #5\\
			Zweitgutachter : #6\\
			\\\
			Abgegeben am \today
  	\end{minipage}
	\end{textblock}
	\	% WICHTIG! Damit wird nach dem Titelblatt eine neue Seite angefangen! Sonst werden Titelblatt &
  	% Danksagung auf eine Seite gedruckt!
}

%---------------------------------------------------------------------------------------------------
% Versicherung über Selbstständigkeit einbinden
%---------------------------------------------------------------------------------------------------
%%---------------------------------------------------------------------------------------------------
% Versicherung über Selbstständigkeit
%---------------------------------------------------------------------------------------------------
\newpage
%\thispagestyle{empty}
\addcontentsline{toc}{section}{Eidesstattliche Erklärung}
\addchap*{Eidesstattliche Erklärung}
\vspace{1cm}
Ich erkläre an Eides statt, dass ich die hier vorgelegte Bachelorthesis selbstständig und ausschließlich unter Verwendung der angegebenen Literatur und sonstigen Hilfsmittel verfasst habe. Die Arbeit wurde in gleicher oder ähnlicher Form keiner anderen Prüfungsbehörde zur Erlangung eines akademischen Grades vorgelegt.
%\vspace{1cm}
\\ \\ \\ \\
\begin{tabularx}{\linewidth}{X l X}
	Karlsruhe, den 31.05.2014	& \qquad \qquad \qquad	& \\
	\cline{1-1}
	\cline{3-3}
	Ort, Datum	& \qquad \qquad \qquad	& Unterschrift (Felix Albert) \\
\end{tabularx}

	\vfill
%	\vfill
%	\vfill
%\pagebreak

\newpage


%---------------------------------------------------------------------------------------------------
% Dient zum Eintragen folgender Dinge in die Zusammenfassung (Abstract):
%	- Thema
% - Stichworte
% - Kurzfassung
% Benutzung wie folgt:
% \abstractentry{Titel}{Text}
%---------------------------------------------------------------------------------------------------
\newcommand{\abstractentry}[2]{
	\textbf{\large#1}\\ 
	\newline
	\nobreakspace 
	\begin{tabular}{lp{142mm}}
		\hspace*{5mm} & #2 \\
	\end{tabular}
	\vfill
}

%---------------------------------------------------------------------------------------------------
% Einbinden von Definition
% Anwendung: \definition{Die Definition}
%---------------------------------------------------------------------------------------------------
%---------------------------------------------------------------------------------------------------
% Erstellt eine Defintion
% Anwendung: \definition{Die Definition}
%---------------------------------------------------------------------------------------------------
\newcommand{\definition}[1]{
\begin{tabular}[ht]{lp{135mm}}
	\textbf{Def.:} & #1 \\
\end{tabular} 
}

%---------------------------------------------------------------------------------------------------
% Einbinden von Widmung
% Anwendung: \dedication{Wem ist das Schriftstück gewidmet}
%---------------------------------------------------------------------------------------------------
%---------------------------------------------------------------------------------------------------
% Erstellt eine Widmung
% Anwendung: \dedication{Wem ist das Schriftst�ck gewidmet}
%---------------------------------------------------------------------------------------------------
\newcommand{\createDedication}[1]{
	\newpage
	\thispagestyle{empty}
	\begin{tabular}{lp{60mm}}
		\hspace*{100mm} & \itshape\rmfamily#1 \\
	\end{tabular}
	\vfill
}

%---------------------------------------------------------------------------------------------------
% Häufig verwendete Namen mit Literaturverweis und Indexeintrag einbinden
%--------------------------------------------------------------------------------------------------
%---------------------------------------------------------------------------------------------------
% H�ufig verwendete Namen mit Literaturverweis und Indexeintrag
%--------------------------------------------------------------------------------------------------
\newcommand{\butrynowski}{Christian Butrynowski\index{Butrynowski, Christian} \citep{Butrynowski:2005}\xspace}
\newcommand{\luepke}{Andr� L�pke\index{L�pke, Andr�}\citep{Luepke:2004}\xspace}
\newcommand{\bresch}{Marco Bresch\index{Bresch, Marco} \citep{Bresch:2004}\xspace}



%---------------------------------------------------------------------------------------------------
% Kürzel einbinden
%---------------------------------------------------------------------------------------------------
% Hier sind Makros definiert, die die Eingabe erleichtern sollen. F�r korrekte Abst�nde zwischen
% "z.B." sorgt also ein "z.\,B." (LaTeX-Befehl f�r kleineren Abstand)
% Schneller schreibt sich das durch das Makro "\zB":

% \newcommand{\zB}{z.\,B.\ }

% Hier ist der Rest aber mit dem Paket xspace verwirklicht. Damit kann
% man bei Bedarf den Abstand mit "\hspace" exakt eingeben. Dann zeigt
% LaTeX keine Toleranz bei den Abk�rzungen und macht eben exakt das
% untenstehende. 

%\renewcommand{\entryname}{K\"urzel}
%\renewcommand{\descriptionname}{Beschreibung}

\newcommand{\eu}{\@\xspace\officialeuro\hspace{0.125em}\@\xspace}
\newcommand{\vgl}{vgl.\@\xspace} 
\newcommand{\abb}{Abb.\@\xspace} 
\newcommand{\tab}{Tab.\@\xspace} 
\newcommand{\zb}{z.\nolinebreak[4]\hspace{0.125em}\nolinebreak[4]B.\@\xspace}
\newcommand{\bzw}{bzw.\@\xspace}
\newcommand{\dahe}{d.\nolinebreak[4]\hspace{0.125em}h.\nolinebreak[4]\@\xspace}
\newcommand{\etc}{etc.\@\xspace}
\newcommand{\s}{S.\@\xspace}
\newcommand{\bzgl}{bzgl.\@\xspace}
\newcommand{\so}{s.\nolinebreak[4]\hspace{0.125em}\nolinebreak[4]o.\@\xspace}
\newcommand{\ia}{i.\nolinebreak[4]\hspace{0.125em}\nolinebreak[4]A.\@\xspace}
\newcommand{\sa}{s.\nolinebreak[4]\hspace{0.125em}\nolinebreak[4]a.\@\xspace}
\newcommand{\su}{s.\nolinebreak[4]\hspace{0.125em}\nolinebreak[4]u.\@\xspace}
\newcommand{\ua}{u.\nolinebreak[4]\hspace{0.125em}\nolinebreak[4]a.\@\xspace}
\newcommand{\og}{o.\nolinebreak[4]\hspace{0.125em}\nolinebreak[4]g.\@\xspace}

\newcommand{\prof}{Prof.\@\xspace}
\newcommand{\dr}{Dr.\@\xspace}

\newcommand{\HSKA}{\myHochschulename \xspace}
\newcommand{\GNU}{GNU\xspace}
\newcommand{\GPL}{\GNU Public License\xspace}

\newcommand{\hpops}{HP Operations Orchestration}

\newcommand{\ACM}{ACM\xspace}
\newcommand{\PDA}{PDA\xspace}

\newcommand{\myLatex}{{\rmfamily\LaTeX\xspace}}		% LaTeX in eigenem Font

%---------------------------------------------------------------------------------------------------
\usepackage{graphicx}                                 % Zur Einbindung von PDF-Bildern
%\usepackage {picins}
\usepackage[colorlinks,															 % Einstellen und Laden des Hyperref-Pakets
	pdftex,
	bookmarks,
	bookmarksopen	=false,
	bookmarksnumbered,
	citecolor		= black, %green, %black,
	linkcolor		= black, %red, %black,
	urlcolor		= black, %blue, %black,
	filecolor		= black,
	linktocpage,
  	pdfstartview	= Fit,                                  % startet mit Ganzseitenanzeige    
	pdfsubject		= {},
	pdftitle		= {Masterthesis},
	pdfauthor		= {Felix Albert}]{hyperref}
\pdfcompresslevel	= 9
 
%---------------------------------------------------------------------------------------------------
% Inhaltsverzeichnis und Abschnittnummerierung
%---------------------------------------------------------------------------------------------------
\setcounter{secnumdepth}{4}   % Bis Ebene 3 erlaubt. (Ebene 4 in Ausnahmefällen).
\setcounter{tocdepth}{4}


%---------------------------------------------------------------------------------------------------
% Kopf- und Fußzeilen
%---------------------------------------------------------------------------------------------------
\pagestyle{scrheadings}
\chapterHeaderWithoutNumbers

%---------------------------------------------------------------------------------------------------
% Trennung
%---------------------------------------------------------------------------------------------------
%---------------------------------------------------------------------------------------------------
% Trennung
% Hier können alle Wörtertrennungen definiert werden. Die nachfolgenden dienen als Beispiel
% und wurden aus der Vorlage von Michael Knop übernommen und um einige durch Ingo Foth ergänzt.
%---------------------------------------------------------------------------------------------------
\hyphenation{In-for-ma-ti-on In-for-ma-ti-on-en Aus-prä-gungs-merk-mal da-rü-ber Web-ap-pli-ka-ti-on Web-ap-pli-ka-tio-nen Web-an-wen-dung Web-an-wen-dung-en My-SQL Kon-text-in-for-ma-ti-onen Flüs-sig-keits-chro-ma-to-gra-fie Dis-tri-bu-ti-on Dis-tri-bu-ti-ons-zen-trum Hard-ware Hard-ware-kom-po-nen-te Hard-ware-kom-po-nen-ten Ma-nage-ment Ma-nage-ment-ap-pli-ka-ti-on Ma-nage-ment-ap-pli-ka-tio-nen Main-frame Main-frames Soft-ware De-si-gner Mes-sage Mes-sage-For-mat Or-ches-trie-rung Ver-gleichs-pha-se Addison Wesley Riley Giarratano Ent-wick-lungs-auf-wand}

%---------------------------------------------------------------------------------------------------
% Anpassung der Parameter, die TeX bei der Berechnung der Zeilenumbrüche verwendet:
%---------------------------------------------------------------------------------------------------
\setlength{\parindent}{0pt}


%\tolerance 1414
%\hbadness 1414
%\emergencystretch 1.5em
%\hfuzz 0.3pt
%\widowpenalty=10000
%\vfuzz \hfuzz
%\raggedbottom
%---------------------------------------------------------------------------------------------------
% Anpassung der Parameter, die TeX bei der Berechnung der Zeilenumbrüche verwendet:
%---------------------------------------------------------------------------------------------------
\tolerance 1414
\hbadness 1414
\emergencystretch 1.5em
\hfuzz 0.3pt
\widowpenalty=10000
\vfuzz \hfuzz
\raggedbottom