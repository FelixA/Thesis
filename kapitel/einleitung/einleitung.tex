%---------------------------------------------------------------------------------------------------
% Einleitung
%---------------------------------------------------------------------------------------------------
%Bsp:
%Für mehrteilige Bücher (Bände 1-5)
%\part{}
%\chapter{}
%Kurzform = anderer Text im Index
%Bsp: \section[Kurzform]{Überschrift}
%\section{}
%\subsection{}
%\subsubsection{}
%Keine Nummer & keine Aufnahme im Index
%\paragraph{}
%\subparagraph{}

\chapter{Einleitung}
\label{chapter_einleitung}


\section{Relevanz der \gls{Blockchain}-Technologie}
Seitdem 2008 von dem nur unter seinem Psudonym Satoshi Nakamoto bekannten Autor erstmals die \gls{Bitcoin}-Architektur beschrieben worden ist, wurde in der Presse vornehmlich der \gls{bitcoin} als Alternativwährung zu etablierten Währungen diskutiert. Mit einer Marktkapitalisierung von knapp 5,7 Mrd USD (Stand 01.02.2016) (Statista) \todo{Zitat}
erfüllt der \gls{bitcoin} vornehmlich eine Funktion als alternative Anlagemöglichkeit zu etablierten Geldanlagen wie Devisen oder Gold. Verschiedene Betrachtungen dieser Crypto-Währungen zeigen \todo{Zitat}, dass zu einer hohen Wahrscheinlichkeit ein großer Teil der Marktkapitalisierung durch Anleger, die den \gls{bitcoin} als Wertanlage statt als Zahlungsmittel interpretieren, entstanden ist.  \\

Zumindest für die zugrundeliegende Technologie könnte sich dies jedoch bald ändern - auch eine Vielzahl anderer Einsatzmöglichkeiten ist möglich \todo{Unbedingt Einleitung ändern}. In der Fachpresse wird mittlerweile die zugrundeliegende Technologie der \gls{Blockchain} - oder Distributed Ledger - als eigentlich disruptive Innovation bezeichnet. Kern der \gls{Blockchain} ist, dass - im Gegensatz zu bisherigen Lösungen - das Hauptbuch nicht in einer zentralen Datenbank, wie z.B. bei einer Bank gespeichert wird, sondern über alle Nutzer verteilt ist. Dadurch wird eine Überweisung von Werten zwischen zwei Parteien nicht von einer zentralen Instanz kontrolliert, sondern von allen Nutzern des Systems. Die wirtschaftlichen Vorteile liegen hierbei auf der Hand - eine zentrale Stelle agiert immer als Dienstleister und verlangt für die Dienstleistung als vertrauenswürdiger Mittelsmann entsprechende Gebühren. Weiterhin kontrolliert sie das System komplett, d.h. bei einem Angriff agiert das System als Single-Point-of-failure. Zusätzlich kann es sich als nicht-vertrauenswürdig herausstellen und selbst die Werte aller Nutzer klauen. Bei verteilten Hauptbüchern entfällt der Bedarf nach einer vertrauenswürdigen dritten Partei, was oben beschriebene Nachteile zumindest reduziert. \\

Die Blockchain-Technologie eignet sich hierbei jedoch nicht nur als Überweisungssystem für Crypto-Währungen. Über sogenannte Smart Contracts, kann eine digitale Münze eine Stellvertreterfunktion für eine Vielzahl verschiedener Einsatzzwecke haben. So kann sie z.B. für ein Wertpapier, eine Domain, eine Identität oder einen reellen Gegenstand stehen. Hierbei wurden bestehende Blockchains - wie z.B. die Bitcoin-Blockchain um verschiedene Funktionalitäten erweitert, als auch komplett neue Entwickelt. \\

Durch die Flexibilität der Einsatzmöglichkeiten kann die Blockchain-Technologie theoretisch überall dort eingesetzt werden, wo bisher ein Vertrag definiert wird, bzw. eine vertrauenswürdige dritte Partei benötigt wird um möglichen Betrug bei einer Transaktion zu verhindern, bzw. als Exekutivkraft eine Umsetzung des Vertrags erzwingt. 

\section{Zielsetzung}
Die Blockchain, die Technologie auf der beispielsweise die Krypto-Währung Bitcoin basiert – verspricht in seiner Urform, durch seine Funktion als dezentrales Hauptbuch – eine sichere, vollautomatisierte Möglichkeit, Werte zwischen verschiedenen Parteien zu übertragen.\\

Seitdem die Technologie das erste Mal beschrieben wurde, haben einige Entwicklungen stattgefunden, die eine Vielzahl weiterer Anwendungszwecke erlauben. Seit einiger Zeit wird der Begriff Blockchain 2.0 für eine teilweise, bzw. vollständig programmierbare Blockchain verwendet. Hierfür wurden entweder bestehende Blockchains (z.B. Bitcoin) erweitert, oder wie im Falle von Ethereum, komplett neue entwickelt.\\
In dieser Arbeit wird untersucht, welche Möglichkeiten die Blockchain-Technologie bietet, aber auch welche Herausforderungen mit ihr verbunden sind. Dazu wird die grundsätzliche Blockchain-Technologie, nach ihrem unter dem Pseudonym Satoshi Nakamoto bekannten Erfinder, beschrieben, sowie die konzeptuellen Änderungen, die eine Turing-vollständige Blockchain – als Einzelentwicklung oder durch Verwendung von Sidechains – ermöglichen, erläutert.\\

Auf Basis dieser Erkenntnisse werden technische Fragestellungen, wie beispielsweise Skalierbarkeit, Kompatibilität zwischen Blockchains, Möglichkeiten zur Anbindung verschiedener Schnittstellen, Sicherheit – insbesondere die Gefahr einer Systemübernahme durch Mining Pools, sowie Möglichkeiten zur Erkennung und zum Auflösen von Manipulationen – diskutiert.\\
Ferner werden die wichtigsten, möglichen Anwendungsbereiche smarter Verträge beschrieben. Deren derzeitige Umsetzbarkeit soll durch eigene Verträge auf Basis einer Turing-vollständigen Blockchain überprüft und bewertet werden.\\

Wichtig sind hier selbstverständlich die bereits beschriebenen technische Fragestellungen, sowie andere nicht-technische: Beispielsweise Schutzmechanismen gegen eine hohe Volatilität der zugrundeliegenden Crypto-Währung, der mögliche Umgang mit fehlerhaften oder betrügerischen Überweisungen, oder die Rolle eines programmierten Vertrags gegenüber eines juristischen.\\

Der Mehrwert der Arbeit stellt eine solide Übersicht über die Praxistauglichkeit einer Technologie dar, die 2008 erstmals beschrieben wurde und deren erste Implementierung als Turing-vollständige Blockchain – die nicht nur das Bitcoin-Protokoll um einen bestimmten Anwendungszweck erweitert – im Juli 2015 veröffentlicht wurde. \\

Leser sollen einen Einblick in die Architektur der Blockchain-Technologie bekommen und das Konzept smarter Verträge und seiner Anwendungsbereiche verstehen. Weiter sollen sie bewerten können, in welchen Situationen der Einsatz der Blockchain-Technologie vorteilhaft gegenüber traditionellen Lösungen ist. \\

Zusätzlich sollen sie beurteilen können, welche Möglichkeiten die Technologie derzeit bietet, welche Nachteile sie mit sich bringt und in welche Richtung sich die weitere Entwicklung bewegen wird.


\section{Aufbau der Arbeit}
%\label{section_motivation}

%---------------------------------------------------------------------------------------------------