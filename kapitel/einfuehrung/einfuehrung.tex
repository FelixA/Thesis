\chapter{Einleitung}
\label{chapter_einleitung}
%---------------------------------------------------------------------------------------------------
% Einführung
%---------------------------------------------------------------------------------------------------
Eine der ausschlaggebenden Erkenntnisse der industriellen Revolution war, dass 
man durch die Definition, Vereinfachung und Standardisierung von Prozessen Arbeit wesentlich effizienter gestalten kann.
\\
\glqq \emph{Dies geschieht durch die Teilung der Arbeit in kleinste Einheiten, zu deren Bewältigung keine oder nur geringe Denkvorgänge zu leisten und die aufgrund des geringen Umfangs bzw. Arbeitsinhalts schnell und repetitiv zu wiederholen sind.}\grqq{} \cite{Maier.2012} \\
%Das hieraus entstandene Scientific Management beruht auf der Annahme, dass sich jeder noch so komplexe Prozess, in einfach ausführbare, simple Teilschritte zerlegen lässt. 
Dieser Ansatz, bekannt unter dem Namen Scientific Management, hat auch Einzug in das Unternehmensumfeld gefunden. \\
\ac{BPM} wird verwendet um Prozesse zu definieren und zu steuern. vgl. \cite{Kiradjiev.} \\
In der Praxis finden sich jedoch immer mehr Prozesse, die sich im Vorfeld nicht definieren lassen, oder deren Ausführung derart Variantenreich ist, dass sie sich mit klassischem \ac{BPM} nicht steuern lassen. Insbesondere Prozesse, die ein hohes Fachwissen der Mitarbeiter benötigen, müssen flexibel Steuerbar sein. Diese Mitarbeiter werden durch \ac{BPM} nicht ausreichend unterstützt und fühlen sich in der Ausführung ihrer Arbeit eingeschränkt.\\
Hier setzt \ac{ACM} an. Im Gegensatz zu \ac{BPM} gibt es hier keine zeitliche und organisatorische Differenz zwischen der Erstellung und der Ausführung eines Prozesses. Die ausführende Person hat die Möglichkeit, Prozesse frei zu gestalten und abzuändern. Lediglich das zu erreichende Oberziel wird vorab definiert. vgl. \cite{Swenson.2012b} \\
Veranschaulichen lässt sich dies am Beispiel eines Navigationsgerätes.
Das Ziel ist bekannt, jedoch kann sich die Route durch Verkehrslage, Wetterlage, Staus oder vorher unbekannten Zwischenzielen immer wieder ändern.
Durch die Übertragung der Verantwortung über einen Prozess an die auszuführende Person ergeben sich jedoch brisante, rechtliche Fragestellungen. Unternehmen haben mittlerweile eine Vielzahl an rechtlichen und internen Richtlinien, welche unter dem Oberbegriff der Compliance zusammengefasst werden. vgl. \cite[s.15]{Walser.2009} \\
Ziel dieser Bachelorarbeit ist, zu ermitteln, inwieweit diese Richtlinien und die Flexibilität eines \ac{ACM}-Systems miteinander vereinbar sind. Hierfür werden gesetzliche Anforderungen ermittelt, überprüft welche Relevant für ein \ac{ACM}-System sind und evaluiert, inwiefern diese für- oder wider dem Merkmal der Flexibilität sprechen. \\
Die Praxistauglichkeit wird überprüft, indem ein realer Vertriebsprozess nach dem Vorbild der Winning-Complex-Sales Methode im Case Manager von IBM implementiert wird. Dies ist eins der bekanntesten \ac{ACM}-Systeme auf dem Markt, so dass die daraus gewonnenen Erkenntnissen Rückschlüsse über den Entwicklungsfortschritt dieser Systeme ziehen lassen.




