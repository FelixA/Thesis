
%---------------------------------------------------------------------------------------------------
\section{Evaluationsmethode}
\label{methode}
Im Folgenden wird die Vorgehensweise bei der Evaluation beschrieben. Diese werden um die Gründe zur Wahl einer qualitativen Methode sowie die dafür verwendeten Ausprägungsmerkmale ergänzt. 

\medskip\noindent Im Rahmen der Evaluation werden alle in \ref{definitionKriterien} definierten Evaluationskriterien mit einem der in Tabelle \ref{Ausprägungsmerkmale} genannten Ausprägungsmerkmale bewertet. Anhand der Einzelbewertungen der Kriterien innerhalb einer in \ref{AnforderungenKategorien} definierten Kategorie, wird eine Gesamtbewertung der Kategorie abgeleitet. Dazu wird eine Annäherung an den Durchschnitt der Einzelbewertungen verwendet. Für das Gesamtsystem wird ebenso verfahren. Entsprechend werden die Evaluationsergebnisse der einzelnen Kategorien als Basis genutzt.
Abschließend werden die Ergebnisse der jeweiligen Systeme miteinander verglichen. Dabei wird auf die Stärken und Schwächen im Vergleich zu den Anderen eingegangen.


\subsection{Gründe für qualitative Methode}
\label{gründeQualitativ}
Das eingangs beschriebene Ziel, die Stärken und Schwächen eines \ac{BPMS} zu ergründen, um damit die Frage  zu klären, ob ein Open Source System eine Alternative darstellen kann, bedingt der Verwendung eines hohen Abstraktionsniveaus. Um die Komplexität eines solch umfassenden Systems in seiner Ganzheitlichkeit erfassen zu können, gibt es zwei Alternativen. Erstere ist die Verwendung eines sehr umfangreichen Kriterienkatalogs, welcher jedes Detail beleuchtet. Eine Verwendung dessen setzt jedoch ein Mindestmaß an Gleichheit bei der Umsetzung der Funktionalitäten jedes der zur Evaluation heran gezogenen Systeme voraus. Die zweite Möglichkeit ist die Verwendung abstrahierter Evaluationskriterien, welche anhand einer subjektiven Meinung, innerhalb der für alle an der Evaluation teilnehmenden gleichen Rahmenbedingungen, bewertet werden. Damit kann sehr viel kontextbezogener auf etwaige Unterschiede eingegangen werden.

%Mehrwert qualitativer Evaluation ->	Fallorientierung
%		Ganzheitlichkeit und Komplexität
%		Kontexte und Hintergründe 
%		Konsistenz und Authentizität
%		Vermeiden verborgener Normativität
%---------------------------------------------------------------------------------------------------

\subsection{Ausprägungsmerkmale}
Zur Evaluation der Systeme wird aufgrund der in \ref{gründeQualitativ} genannten Vorteile eine qualitative Methode verwendet. Diese verwendet eine Skala mit fünf Ausprägungsmerkmalen für die Beschreibung der Fähigkeiten des zu evaluierenden Systems im betreffenden Evaluationskriterium. Die Einordnung in ein Ausprägungsmerkmal erfolgt anhand einer subjektiven Bewertung. Die Bedingungen zur Einteilung in eines der Ausprägungsmerkmale können der Tabelle \ref{Ausprägungsmerkmale} entnommen werden.

\begin{table}[!ht]
\small
\caption{Ausprägungsmerkmale der Bewertungsmethode}
\label{Ausprägungsmerkmale}
\begin{tabularx}{\textwidth}[b]{|p{2.5cm}|X|}
\hline
\rowcolor{black!10} \centering \normalsize\textbf{Ausprägung} & \normalsize\textbf{Beschreibung}  \\
\hline
\centering\arraybackslash \textcircled{-} \textcircled{-} & Die Funktionalität ist nicht vorhanden oder derart fehlerhaft, dass diese nicht verwendbar ist. \\
\hline
\centering\arraybackslash \textcircled{-} & Die Funktionalität ist nur unzureichend vorhanden, d.h. sie ist sehr umständlich zu verwenden oder weist grobe Mängel bzw. Fehler auf. \\
\hline
\centering\arraybackslash \textcircled{} & Die Funktionalität ist prinzipiell vorhanden, weist jedoch Fehler bzw. Unzulänglichkeiten auf oder ist nicht intuitiv benutzbar. \\
\hline
\centering\arraybackslash \textcircled{+} & Die Funktionalität umfasst alle wichtigen Teile, weist keine Fehler auf, welche die Benutzbarkeit einschränken, und lässt sich nachvollziehbar bedienen. \\
\hline
\centering\arraybackslash \textcircled{+} \textcircled{+} & Die Funktionalität ist vollständig implementiert und lässt sich intuitiv benutzen. \\
\hline
\end{tabularx}
\end{table}
\normalsize

%---------------------------------------------------------------------------------------------------
\section{Voraussetzungen für Open Source Lösung als Alternative}
\label{voraussetzungAlternative}
Um dem erklärten Ziel nachzukommen, in welcher Hinsicht ein Open Source BPMS als Alternative zu den am Markt etablierten proprietären Lösungen gesehen werden kann, werden die erreichten Punkte der einzelnen Bewertungskriterien in Kategorien zusammengefasst und aufsummiert. Dabei sollte das Open Source BPMS in keiner Kategorie gravierende Mängel oder Lücken aufweisen und nicht deutlich hinter den erreichten Werten der proprietären BPMS liegen. Eine Lücke bzw. ein gravierender Mangel liegt vor, wenn die geforderte Funktionalität nicht vorhanden ist, oder derart unvollständig implementiert ist, dass eine Benutzung derer unter realistischen Umständen als nicht möglich angesehen werden kann. 

%---------------------------------------------------------------------------------------------------

