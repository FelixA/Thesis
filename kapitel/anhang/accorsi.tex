\section{Anforderungen an adaptive BPMS nach Accorsi}
\begin{enumerate} %[\itshape]
\item \textbf{Autorisierung} regelt, welche Subjekte oder Rollen auf welche Ressourcen zugreifen können. In betrieblichen IT-Landschaften wird dies oft anhand des rollenbasierten Zugriffskontrollmodells formalisiert. Solche Modelle können selbst in kleinen Unternehmen bzw. Systemen über komplexe, hochdynamische Rollenmodelle mit über 500 Rollen verfügen.
 \item \textbf{Nutzungskontrolle} drückt Bedingungen aus, die nach der Autorisierung des Zugriffs auf eine Ressource deren Nutzung regeln. So kann zum Beispiel die maximale Zugriffszahl auf ein Datenobjekt festgelegt werden oder die Verpflichtung, dass jedem Zugriff auf ein Datum die entsprechende Löschoperation folgen muss. Generell beziehen sich diese Anforderungen auf das Zusammenspiel mehrerer Aktivitäten im Prozess (sog. Kontrollfluss), welche als Patterns erfasst werden können.
 \item \textbf{Interessenkonflikt} zielt darauf ab, unzulässige Ausnutzung von Insiderwissen zu unterbinden. Erfasst werden solche Anforderungen anhand des Chinese Wall-Modells in Anlehnung an die Metapher, dass objekt- und subjektspezifische Zugriffsbeschränkungen wie Mauern um Schutzobjekte aufgebaut werden. 
 \item \textbf{Funktionstrennung} bedeutet, dass bestimmte Aufgaben in einem Geschäftsprozess nicht durch ein und dieselbe Organisationseinheit (z.B. Subjekt, Rolle oder Abteilung) durchgeführt werden sollen. Diese Kontrolle soll kriminelle Handlungen und Betrug unterbinden.
 \item \textbf{Aufgabenbindung} bezeichnet das Gegenteil von Funktionstrennung. Dies verlangt, dass bestimmte Aufgaben in einem Geschäftsprozess durch eine Organisationseinheit durchgeführt werden müssen.
 \item \textbf{Mehr-Augen-Kontrolle} besagt, dass eine kritische Aktivität im Prozess nicht von einer einzelnen Person durchgeführt werden soll. Ziel dieser eng mit der Funktionstrennung verwandten Kontrolle ist es, den Eintritt von Fehlern und Missbrauch zu reduzieren.
 \item \textbf{Isolation} drückt aus, dass innerhalb eines oder zwischen mehreren Prozessen keine Informations- oder Datenlecks auftreten können. Damit soll die Vertraulichkeit von Informationen oder Daten gewährleistet werden.
 \end{enumerate}